\chapter{Podsumowanie}
\thispagestyle{chapterBeginStyle}



\section{Aplikacja i praca}
    Aplikacja przedstawiona w tej pracy umożliwia rekreacyjne rozwiązywanie predefiniowanych
nonogramów oraz zapewnia dostęp do szybkiego solvera nonogramów. Dzięki zaimplementowaniu programu
na urządzenia mobilne, korzystanie z tych funkcjonalności jest możliwe praktycznie wszędzie.

    W pracy zostały opisane pojęcia teoretyczne związane z rozwiązywaniem nonogramów, 
między innymi klasy złożoności problemów. Nakreślona została złożoność tej klasy łamigłówek 
oraz pokazane zostały różne podejścia do ich rozwiązywania. Porównanie algorytmów rozwiązujących
nonogramy oraz zbadanie wpływu heurystyk na ich wydajność pozwala lepiej zrozumieć istotę
złożoności danych plansz i umożliwić łatwiejsze określenie ich trudności.



\section{Możliwości rozwoju}
    Dla aplikacji przedstawionej w pracy istnieje szereg funkcjonalności, które zwiększyłyby jej
atrakcyjność i konkurencyjność.

\subsubsection{Automatyczne generowanie łamigłówek}
    Dzięki automatycznemu generowaniu łamigłówek o zadanych parametrach, aplikacja zyskałaby
na powtarzalności. Użytkownik wybierałby trudność i rozmiar łamigłówki, a moduł zwracałby
wygenerowaną łamigłówkę do rozwiązania. Wyzwaniem przy tworzeniu tej funkcjonalności byłoby takie
dobranie parametrów, by użytkownik mógł oczekiwać nonogramów mieszczących się w wybranym zakresie
trudności. Algorytm mógłby ocenić trudność wygenerowanej łamigłówki poprzez sprawdzenie ilości liczb
we wskazówkach, ilości wypełnionych pól w liniach, ruchów potrzebnych solverowi do jej rozwiązania,
bądź konieczności zakładania (lub jej braku).

\subsubsection{Sczytywanie nonogramów ze zdjęć}
    Implementacja sczytywania nonogramów ze zdjęć w znaczącym stopniu ułatwiłaby wprowadzanie
łamigłówek dla solvera. Dla tego modułu konieczne byłoby opracowanie rozpoznawania łamigłówek w taki
sposób, by układ wskazówek i liczb we wskazówkach pokrywał się z siatką, na której zdefiniowana jest
łamigłówka. Modyfikacja polegająca na rozpoznawaniu już wypełnionych pól na planszy jest
także warta rozważenia.

\subsubsection{Tworzenie łamigłówek dla innych graczy}
    Podobnie jak w przypadku automatycznego generowania łamigłówek, dzielenie się łamigłówkami
znacznie zwiększyłoby powtarzalność aplikacji. Jednak w odróżnieniu od wyżej wymienionej funkcjonalności,
łamigłówki tworzone przez graczy mogłyby przedstawiać rzeczywiste obiekty, podobnie jak jest w przypadku
łamigłówek dostępnych w aplikacji. Takie łamigłówki miałyby pewne walory estetyczne. Dodanie
tego modułu wiązałoby się z implementacją serwera, który zbierałby łamigłówki od różnych graczy.
Aplikacja powinna także zostać poszerzona o możliwość oceny rozwiązanych łamigłówek.

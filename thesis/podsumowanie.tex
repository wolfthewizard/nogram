\chapter{Podsumowanie}
\thispagestyle{chapterBeginStyle}



\section{Aplikacja i praca}
    Aplikacja przedstawiona w tej pracy ma umożliwić zarówno rekreacyjne rozwiązywanie predefiniowanych
nonogramów, jak i zapewnić dostęp do szybkiego solvera nonogramów. 

    W pracy zostały opisane pojęcia teoretyczne związane z rozwiązywaniem nonogramów, 
między innymi klasy złożoności problemów. Nakreślona została złożoność tej klasy łamigłówek 
oraz pokazane zostały różne podejścia do ich rozwiązywania. Porównanie algorytmów rozwiązujących
nonogramy oraz zbadanie wpływu heurystyk na ich wydajność pozwala lepiej zrozumieć istotę
złożoności danych plansz i umożliwić łatwiejsze określenie ich trudności.


\section{Możliwości rozwoju}
    Aplikacja może zostać poszerzona o moduł generujący łamigłówki żądanej trudności. Dzięki temu
program zyskałby powtarzalność i umożliwił użytkownikowi dobranie odpowiedniego wyzwania na
daną chwilę. Dodanie modułu sczytującego plansze ze zdjęć znacznie ułatwiłoby wprowadzanie łamigłówek
dla solvera. W dłuższej perspektywie, aplikacja skorzystałaby z dodania funkcji publikacji
łamigłówek i rozwiązywania łamigłówek stworzonych przez innych graczy.

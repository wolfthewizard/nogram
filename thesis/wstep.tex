%Korekta ALD - nienumerowany wstęp
%\chapter{Wstęp}
\addcontentsline{toc}{chapter}{Wstęp}
\chapter*{Wstęp}

\thispagestyle{chapterBeginStyle}

	Praca poświęcona jest aplikacji rozwiązującej nonogramy. W pracy opisane są zagadnienia
teoretyczne potrzebne do zrozumienia problemu, układ i sposób użycia samej aplikacji, a także
porównanie algorytmów rozwiązujących nonogramy (określanych dalej jako "solvery").

	Celem pracy jest stworzenie aplikacji łączącej dwie główne funkcjonalności: udostępnienie
użytkownikowi zestawu łamigłówek do rozwiązania oraz zapewnienie dostępu do solvera, dzięki
któremu będzie mógł zobaczyć rozwiązanie napotkanych łamigłówek. Dodatkowym wymogiem jest łatwy
dostęp do obydwu wymienionych funkcjonalności.

	Aplikacje dostępne na rynku nie spełniają obu tych roli równocześnie. Programy dedykowane dla
urządzeń mobilnych nie umożliwiają sprawdzenia rozwiązań nonogramów za pomocą algorytmu
zintegrowanego w aplikację. Rozwiązania dostępne na komputerach osobistych dzielą się na te udostępniające
dostęp do solvera i na takie, które oferują zestawy łamigłówek do rozwiązania. Brakuje jednak
aplikacji, która łączy te dwie funkcjonalności, i jest łatwo dostępna — poprzez urządzenie mobilne.

	Praca podzielona jest na 6 części.

	W pierwszej części opisany jest problem rozwiązywania nonogramów. Wytłumaczone jest
czym są nonogramy. Podane są definicje potrzebne do zrozumienia problemu. Na końcu, za pomocą
opisanych zagadnień, wyznaczona jest trudność problemu.

	Druga część zawiera schematyczny opis aplikacji — listę wymagań spełnianych przez
aplikację oraz zarys nawigacji po niej.

	W trzecim rozdziale zawarte są informacje dotyczące implementacji: użyte technologie wraz
z ich opisem oraz schemat wykorzystanej bazy danych.

	Czwarta część skupia się na opisie solvera wykorzystanego w aplikacji. Nakreślona jest ewolucja
solvera w trakcie rozwoju aplikacji, zaproponowane są także heurystyki mogące usprawnić działanie
solvera. Różne implementacje wraz z usprawnieniami są poddane testom, by ocenić ich wydajność,
oraz podane są wnioski wyciągnięte z badań. Dodatkowo, nakreślone są wymagania spełniane przez
łamigłówki dodane do aplikacji.

	Piąty rozdział to krótka instrukcja dla użytkownika, opisująca wymagania do uruchomienia aplikacji.

	Na końcu, w szóstej części, zawarte jest podsumowanie pracy oraz kierunki, w których
można dokonać dalszego rozwoju aplikacji.

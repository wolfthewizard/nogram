\chapter{Wstęp}
\thispagestyle{chapterBeginStyle}



	Praca poświęcona jest aplikacji rozwiązującej nonogramy. W pracy opisane są zagadnienia
teoretyczne potrzebne do zrozumienia problemu, układ i sposób użycia samej aplikacji, a także
porównanie podejść do rozwiązywania nonogramów.

	Praca podzielona jest na 7 części.

	Pierwszą część stanowi wstęp, wprowadzający odbiorcę w tematykę pracy.

	W drugiej części opisany jest problem rozwiązywania nonogramów. Wytłumaczone jest
czym są nonogramy. Podane są definicje potrzebne do zrozumienia problemu. Na końcu, za pomocą
opisanych zagadnień wyznaczona jest trudność problemu.

	Trzecia część zawiera schematyczny opis aplikacji - listę wymagań spełnianych przez
aplikację oraz zarys nawigacji po niej.

	W czwartym rozdziale zawarte są informacje dotyczące implementacji: użyte technologie wraz
z ich opisem oraz schemat wykorzystanej bazy danych.

	Piąta część skupia się na opisie solvera wykorzystanego w aplikacji. Nakreślona jest ewolucja
solvera w trakcie rozwoju aplikacji, zaproponowane są także heurystyki mogące usprawnić działanie
solvera. Różne implementacje wraz z usprawnieniami są poddane testom, by ocenić ich wydajność,
oraz podane są wnioski wyciągnięte z badań.

	Szósty rozdział to krótka instrukcja dla użytkownika, opisująca wymagania do uruchomienia aplikacji.

	Na końcu, w siódmej części, zawarte jest podsumowanie pracy oraz kierunki w których
można dokonać dalszego rozwoju aplikacji.

% ***************************************************************************************************
%
%	Szablon pracy magisterskiej dla Politechniki Wrocławskiej w wersji dwustronnej.
%	Autor:	Tomasz Strzałka
% Koretkta i dostosowanie do wymogów WIT 3.12.2021: dr inż. Anna Lauks-Dutka
%
% ***************************************************************************************************

% Styl dwustronny z domyślną wielkością czcionki 10pt oraz oddzieloną stroną tytułową (titlepage).
% Domyślnie rodziały rozpoczynają się na stronie prawej (openright).
\documentclass[10pt]{book}
\usepackage{times}


% ***************************************************************************************************
% Ustawienia języka
% ***************************************************************************************************

% Podstawowe ustawienia języka, według którego formatowany będzie dokument
\usepackage[polish]{babel}

% Pakiet babel dla polskiego języka powoduje konflikt z pakietem amssymb.
% Polecenie '\lll' definiują oba pakiety - porządana jest druga definicja.
\let\lll\undefined

% W przypadku wielojęzykowości ustawia główny język dokumentu
\selectlanguage{polish}

% Kodowanie dokumentu
\usepackage[utf8]{inputenc}

% Dowolny rozmiar czcionek, kodowanie znaków
\usepackage{lmodern}

% Polskie wcięcia akapitów
\usepackage{indentfirst}

% Polskie łamanie wyrazów
\usepackage[plmath]{polski}

% Przecinek w wyrażeniach matematycznych zamiast kropki
\usepackage{icomma}

% Polskie formatowanie typograficzne
\frenchspacing

% Zapewnia liczne usprawnienia wyświetlania i organizacji matematycznych formuł. 
\usepackage{amsmath}

% Wprowadza rozszerzony zestaw symboli m.in. \leadsto
\usepackage{amssymb}

% Dodatkowa, ,,kręcona'' czcionka matematyczna
\usepackage{mathrsfs}

% Dodatkowe wsparcie dla środowiska mathbb, które nie wspiera domyślnie cyfr (\mathbb{})
\usepackage{bbold}

% Fixes/improves amsmath
\usepackage{mathtools}


% ***************************************************************************************************
% Kolory  
% ***************************************************************************************************

% Umożliwia kolorowanie poszczególnych komórek tabeli
\usepackage[table]{xcolor}% http://ctan.org/pkg/

% Umożliwia łatwą zmianę koloru linii w tabeli
\usepackage{tabu}

% Umożliwia rozszerzoną kontrolę nad kolorami.
\usepackage{xcolor}

% Definicje kolorów
\definecolor{lgray}{HTML}{9F9F9F}
\definecolor{dgray}{HTML}{5F5F5F}
% lgray				-	nazwa nowo zdefiniowanego koloru
% HTML				-	model kolorów
% CCCCCC			-	wartość koloru zgodna z modelem

% ***************************************************************************************************
% Algorytmy 
% ***************************************************************************************************

% Udostępnia środowisko do konstruowania pseudokodów
\usepackage[ruled,vlined,linesnumbered,longend,algochapter]{algorithm2e}
% ruled	- poziome kreski na początku i końcu algorytmu, podpis na górze oddzielony również kreską poziomą
% vlined - pionowe kreski łączące początek polecenia z jego końcem
% linesnumbered	- numerowanie kolejnych wierszy algorytmu
% longend - długie końcówki np. ifend, forend itd.
% algochapter - numeracja z rozdziałami

% Zamiana nazwy środowiska z domyślnej "Algorithm X" na "Pseudokod X"
\newenvironment{pseudokod}[1][htb]{
	\renewcommand{\algorithmcfname}{Pseudokod}
	\begin{algorithm}[#1]%
	}{
\end{algorithm}
}

% Zmiana rozmiaru komentarzy
\newcommand\algcomment[1]{
	\footnotesize{#1}
}

% Ustawienie zadanego stylu dla komentarzy
\SetCommentSty{algcomment}

% Wyśrodkowana tylda
\usepackage{textcomp}%
\newcommand{\textapprox}{\raisebox{0.5ex}{\texttildelow}}

% Listowanie kodów źródłowych
\usepackage{listings} 
\renewcommand{\lstlistingname}{Kod źródłowy} % Polska nazwa listingu

% Definicje pecjalnych znaków, które nie są obsługiwane w środowisku listing
\lstset{literate=
	{ż}{{\.{z}}}1	{ź}{{\'{z}}}1
	{ć}{{\'{c}}}1	{ń}{{\'{n}}}1
	{ą}{{\c a}}1	{ś}{{\'{s}}}1
	{ł}{{\l}}1		{ę}{{\c{e}}}1
	{ó}{{\'{o}}}1	{á}{{\'a}}1
	{é}{{\'e}}1		{í}{{\'i}}1
	{ó}{{\'o}}1		{ú}{{\'u}}1
	{ù}{{\`u}}1		{Á}{{\'A}}1
	{É}{{\'E}}1		{Í}{{\'I}}1
	{Ó}{{\'O}}1		{Ú}{{\'U}}1
	{à}{{\`a}}1		{è}{{\'e}}1
	{ì}{{\`i}}1		{ò}{{\`o}}1
	{ò}{{\`o}}1		{À}{{\`A}}1
	{È}{{\'E}}1		{Ì}{{\`I}}1
	{Ò}{{\`O}}1		{Ò}{{\`O}}1
	{ä}{{\"a}}1		{ë}{{\"e}}1
	{ï}{{\"i}}1		{ö}{{\"o}}1
	{ü}{{\"u}}1		{Ä}{{\"A}}1
	{Ë}{{\"E}}1		{Ï}{{\"I}}1
	{Ö}{{\"O}}1		{Ü}{{\"U}}1
	{â}{{\^a}}1		{ê}{{\^e}}1
	{î}{{\^i}}1		{ô}{{\^o}}1
	{û}{{\^u}}1		{Â}{{\^A}}1
	{Ê}{{\^E}}1		{Î}{{\^I}}1
	{Ô}{{\^O}}1		{Û}{{\^U}}1
	{œ}{{\oe}}1		{Œ}{{\OE}}1
	{æ}{{\ae}}1		{Æ}{{\AE}}1
	{ß}{{\ss}}1		{ç}{{\c c}}1
	{Ç}{{\c C}}1	{ø}{{\o}}1
	{å}{{\r a}}1	{Å}{{\r A}}1
	{€}{{\EUR}}1	{£}{{\pounds}}1
}

% ***************************************************************************************************
% Marginesy 
% ***************************************************************************************************

% Ustawienia rozmiarów stron i ich marginesów
%korekta ALD - dodatkowe 0.5cm na oprawę z lewej
%\usepackage[headheight=18pt, top=25mm, bottom=25mm, left=25mm, right=25mm]{geometry}
\usepackage[headheight=18pt, top=25mm, bottom=25mm, left=30mm, right=25mm]{geometry}
% headheight		-	wysokość tytułów
% top				-	margines górny
% bottom			-	margines dolny
% left				-	margines lewy
% right				-	margines prawy

% Usunięcie górnego marginesu dla środowisk
\makeatletter
\setlength\@fptop{0\p@}	
\makeatother

% ***************************************************************************************************
% Styl 
% ***************************************************************************************************

% Definiuje środowisko 'titlingpage', które zapewnia pełną kontrolę nad układem strony tytułowej.
\usepackage{titling}


% Umożliwia modyfikowanie stylu spisu treści
\usepackage{tocloft}	

\tocloftpagestyle{tableOfContentStyle}

% Definiowanie własnych stylów nagłówków i/lub stopek
\usepackage{fancyhdr}

% Domyślny styl dla pracy 
\fancypagestyle{custom}{
	\fancyhf{}									% wyczyść stopki i nagłówki
	\fancyhead[RO]{								% Prawy, nieparzysty nagłówek
%korekta ALD: 
%\hrulefill \hspace{16pt} \large Rozdział \thechapter
		\hrulefill \hspace{16pt} \large \ifnum \thechapter>0 {Rozdział \thechapter} \else{Wstęp}\fi
		\put(-472.1, 12.1){%
			\makebox(0,0)[l]{%
                \includegraphics[width=0.05\textwidth]{pwr-logo}
			}
		}
		\put(-443,5.5){%
			\makebox(0,0)[l]{%
				\small Politechnika Wrocławska
			}
		}
	}
	\fancyhead[LE]{								% Lewy, parzysty nagłówek
%korekta ALD: 
%\large Rozdział \thechapter \hspace{16pt} \hrulefill 
        \large \ifnum \thechapter>0 {Rozdział \thechapter} \else{Wstęp}\fi \hspace{16pt} \hrulefill 
		\put(-22, 12.1){%
			\makebox(0,0)[l]{%
                %korekta ALD
				%\includegraphics[width=0.05\textwidth]{pwr-logo}
                \includegraphics[width=0.05\textwidth]{wit-logo}			}
		}
		\put(-210,5.5){%
			\makebox(0,0)[l]{%
%				\small Wydział Podstawowych Problemów Techniki
%Korekta ALD
\small Wydział Informatyki i Telekomunikacji
			}
		}
	}
	\fancyfoot[LE,RO]{							% Stopki
		\thepage
	}
	\renewcommand{\headrulewidth}{0pt}			% Grubość linii w nagłówku
	\renewcommand{\footrulewidth}{0.2pt}		% Grubość linii w stopce
}


% Domyślny styl dla bibliografii
\fancypagestyle{bibliographyStyle}{
	\fancyhf{}									% wyczyść stopki i nagłówki
	\fancyhead[RO]{								% Prawy, nieparzysty nagłówek
		\hrulefill \hspace{16pt} \large Dodatek \thechapter
		\put(-472.1, 12.1){%
			\makebox(0,0)[l]{%
	\includegraphics[width=0.05\textwidth]{pwr-logo}
			}
		}
		\put(-443,5.5){%
			\makebox(0,0)[l]{%
				\small Politechnika Wrocławska
			}
		}
	}
	\fancyhead[LE]{								% Lewy, parzysty nagłówek
		\large Bibliografia \hspace{16pt} \hrulefill 
		\put(-22, 12.1){%
			\makebox(0,0)[l]{%
			%korekta ALD	\includegraphics[width=0.05\textwidth]{wppt-logo}
			\includegraphics[width=0.05\textwidth]{wit-logo}
			}
		}
		\put(-210,5.5){%
			\makebox(0,0)[l]{%
			%korekta ALD
				%\small Wydział Podstawowych Problemów Techniki
				\small Wydział Informatyki i Telekomunikacji
			}
		}
	}
	\fancyfoot[LE,RO]{							% Stopki
		\thepage
	}
	\renewcommand{\headrulewidth}{0pt}			% Grubość linii w nagłówku
	\renewcommand{\footrulewidth}{0.2pt}		% Grubość linii w stopce
}

% Domyślny styl dla spisu tabel i rysunków
\fancypagestyle{listOfTablesStyle}{
	\fancyhf{}									% wyczyść stopki i nagłówki
	\fancyhead[RO]{								% Prawy, nieparzysty nagłówek
		\hrulefill \hspace{16pt} \large Spis tabel
		\put(-472.1, 12.1){%
			\makebox(0,0)[l]{%
			\includegraphics[width=0.05\textwidth]{pwr-logo}
			}
		}
		\put(-443,5.5){%
			\makebox(0,0)[l]{%
				\small Politechnika Wrocławska
			}
		}
	}
	\fancyhead[LE]{								% Lewy, parzysty nagłówek
		\large Spis tabel \hspace{16pt} \hrulefill 
		\put(-22, 12.1){%
			\makebox(0,0)[l]{%
\includegraphics[width=0.05\textwidth]{wit-logo}
			}
		}
		\put(-210,5.5){%
			\makebox(0,0)[l]{%
				\small Wydział Informatyki i Telekomunikacji
			}
		}
	}
	\fancyfoot[LE,RO]{							% Stopki
		\thepage
	}
	\renewcommand{\headrulewidth}{0pt}			% Grubość linii w nagłówku
	\renewcommand{\footrulewidth}{0.2pt}		% Grubość linii w stopce
}

\fancypagestyle{listOfPlotsStyle}{
	\fancyhf{}									% wyczyść stopki i nagłówki
	\fancyhead[RO]{								% Prawy, nieparzysty nagłówek
		\hrulefill \hspace{16pt} \large Spis rysunków
		\put(-472.1, 12.1){%
			\makebox(0,0)[l]{%
			\includegraphics[width=0.05\textwidth]{pwr-logo}
			}
		}
		\put(-443,5.5){%
			\makebox(0,0)[l]{%
				\small Politechnika Wrocławska
			}
		}
	}
	\fancyhead[LE]{								% Lewy, parzysty nagłówek
		\large Spis rysunków \hspace{16pt} \hrulefill 
		\put(-22, 12.1){%
			\makebox(0,0)[l]{%
\includegraphics[width=0.05\textwidth]{wit-logo}
			}
		}
		\put(-210,5.5){%
			\makebox(0,0)[l]{%
				\small Wydział Informatyki i Telekomunikacji
			}
		}
	}
	\fancyfoot[LE,RO]{							% Stopki
		\thepage
	}
	\renewcommand{\headrulewidth}{0pt}			% Grubość linii w nagłówku
	\renewcommand{\footrulewidth}{0.2pt}		% Grubość linii w stopce
}

% Domyślny styl dla dodatków
\fancypagestyle{appendixStyle}{
	\fancyhf{}									% wyczyść stopki i nagłówki
	\fancyhead[RO]{								% Prawy, nieparzysty nagłówek
		\hrulefill \hspace{16pt} \large Załącznik \thechapter
		\put(-472.1, 12.1){%
			\makebox(0,0)[l]{%
\includegraphics[width=0.05\textwidth]{pwr-logo}
			}
		}
		\put(-443,5.5){%
			\makebox(0,0)[l]{%
				\small Politechnika Wrocławska
			}
		}
	}
	\fancyhead[LE]{								% Lewy, parzysty nagłówek
		\large Załącznik \thechapter \hspace{16pt} \hrulefill 
		\put(-22, 12.1){%
			\makebox(0,0)[l]{%
%korekta ALD:				\includegraphics[width=0.05\textwidth]{wppt-logo}
\includegraphics[width=0.05\textwidth]{wit-logo}
			}
		}
		\put(-210,5.5){%
			\makebox(0,0)[l]{%
			%korekta ALD
				%\small Wydział Podstawowych Problemów Techniki
				\small Wydział Informatyki i Telekomunikacji
			}
		}
	}
	\fancyfoot[LE,RO]{							% Stopki
		\thepage
	}
	\renewcommand{\headrulewidth}{0pt}			% Grubość linii w nagłówku
	\renewcommand{\footrulewidth}{0.2pt}		% Grubość linii w stopce
}

% Osobny styl dla stron zaczynających rozdział/spis treści itd. (domyślnie formatowane jako "plain")
\fancypagestyle{chapterBeginStyle}{
	\fancyhf{}%
	\fancyfoot[LE,RO]{
		\thepage
	}
	\renewcommand{\headrulewidth}{0pt}
	\renewcommand{\footrulewidth}{0.2pt}
}

% Styl dla pozostałych stron spisu treści
\fancypagestyle{tableOfContentStyle}{
	\fancyhf{}%
	\fancyfoot[LE,RO]{
		\thepage
	}
	\renewcommand{\headrulewidth}{0pt}
	\renewcommand{\footrulewidth}{0.2pt}
}
% Formatowanie tytułów rozdziałów i/lub sekcji
\usepackage{titlesec}

% Formatowanie tytułów rozdziałów
\titleformat{\chapter}[hang]					% kształt
{
	\vspace{-10ex}
	%\Huge
	\large
	\bfseries
}												% formatowanie tekstu modyfikowanego elementu
{}												% etykieta występująca przed tekstem modyfikowanego elementu, niewidoczna w spisie treści
{
	10pt
}												% odstęp formatowanego tytułu od lewego marginesu/etykiety
{
    \large
	\bfseries
}												% formatowanie elementów przed modyfikowanym tytułem
[
\vspace{2ex}
%\rule{\textwidth}{0.4pt}
%\vspace{-4ex}
]												% dodatkowe formatowanie stosowane poniżej modyfikowanego tytułu


% Formatowanie tytułów sekcji
\titleformat{\section}[hang]					% kształt
{
	\vspace{2ex}
%	\titlerule\vspace{1ex}
	\large\bfseries
}												% formatowanie tekstu modyfikowanego elementu
{
	\thesection									% etykieta występująca przed tekstem modyfikowanego elementu, niewidoczna w spisie treści
}
{
	0pt
}												% odstęp formatowanego tytułu od lewego marginesu/etykiety
{
	\large
	\bfseries
}												% formatowanie elementów przed modyfikowanym tytułem


%ALD- ustawienia wielkości fontów dla rozdziałów i sekcji
\usepackage{sectsty}
%\chapterfont{\fontsize{14}{17.6}\selectfont}
\sectionfont{\fontsize{13}{16.8}\selectfont}
\subsectionfont{\fontsize{12}{15.6}\selectfont}

% ***************************************************************************************************
% Linki
% ***************************************************************************************************

% Umożliwia wstawianie hiperłączy do dokumentu
\usepackage{hyperref}							% Aktywuje linki

\hypersetup{
	colorlinks	=	true,					% Koloruje tekst zamiast tworzyć ramki.
	linkcolor		=	blue,					% Kolory: referencji,
        citecolor		=	blue,					% cytowań,
	urlcolor		=	blue					% hiperlinków.
}

% Do stworzenia hiperłączy zostanie użyta ta sama (same) czcionka co dla reszty dokumentu
\urlstyle{same}




% ***************************************************************************************************
% Linki
% ***************************************************************************************************

% Umożliwia zdefiniowanie własnego stylu wyliczeniowego
\usepackage{enumitem}

% Nowa lista numerowana z trzema poziomami
\newlist{myitemize}{itemize}{3}

% Definicja wyglądu znacznika pierwszego poziomu
\setlist[myitemize,1]{
	label		=	\textbullet,
	leftmargin	=	4mm}

% Definicja wyglądu znacznika drugiego poziomu
\setlist[myitemize,2]{
	label		=	$\diamond$,
	leftmargin	=	8mm}

% Definicja wyglądu znacznika trzeciego poziomu
\setlist[myitemize,3]{
	label		=	$\diamond$,
	leftmargin	=	12mm
}

% ***************************************************************************************************
% Inne pakiety
% ***************************************************************************************************

% Dołączanie rysunków
\usepackage{graphicx}

% Figury i przypisy
\usepackage{caption}
\usepackage{subcaption}

% Umożliwia tworzenie przypisów wewnątrz środowisk
\usepackage{footnote}

% Umożliwia tworzenie struktur katalogów
\usepackage{dirtree}

% Rozciąganie komórek tabeli na wiele wierszy
\usepackage{multirow}

% Precyzyjne obliczenia szerokości/wysokości dowolnego fragmentu wygenerowanego przez LaTeX
\usepackage{calc}

% ***************************************************************************************************
% Matematyczne skróty
% ***************************************************************************************************

% Skrócony symbol liczb rzeczywistych
\newcommand{\RR}{\mathbb{R}}

% Skrócony symbol liczb naturalnych
\newcommand{\NN}{\mathbb{N}}

% Skrócony symbol liczb wymiernych
\newcommand{\QQ}{\mathbb{Q}}

% Skrócony symbol liczb całkowitych
\newcommand{\ZZ}{\mathbb{Z}}

% Skrócony symbol logicznej implikacji
\newcommand{\IMP}{\rightarrow}

% Skrócony symbol  logicznej równoważności
\newcommand{\IFF}{\leftrightarrow}

% ***************************************************************************************************
% Środowiska
% ***************************************************************************************************

% Środowisko do twierdzeń
\newtheorem{theorem}{Twierdzenie}[chapter]

% Środowisko do lematów
\newtheorem{lemma}{Lemat}[chapter]

% Środowisko do przykładów
\newtheorem{example}{Przykład}[chapter]

% Środowisko do wniosków
\newtheorem{corollary}{Wniosek}[chapter]

% Środowisko do definicji
\newtheorem{definition}{Definicja}[chapter]

% Środowisko do dowodów
\newenvironment{proof}{
	\par\noindent \textbf{Dowód.}
}{
\begin{flushright}
	\vspace*{-6mm}\mbox{$\blacklozenge$}
\end{flushright}
}

%ALD - nowe środowisko do streszczenia i abstractu
\newenvironment{streszczenie}{
	\par\noindent {\large \textbf{Streszczenie}\\[14pt]\indent}
}{}
\newenvironment{abstract}{
	\par\noindent {\large \textbf{Abstract}\\[14pt]\indent}
}{}

% Środowisko do uwag
\newenvironment{remark}{
	\bigskip \par\noindent \small \textbf{Uwaga.}
}{
\begin{small}
	\vspace*{4mm}
\end{small}
}

% ***************************************************************************************************
% Słownik
% ***************************************************************************************************

% Prawidłowe dzielenie wyrazów
\hyphenation{wszy-stkich ko-lu-mnę każ-da od-leg-łość
	dzie-dzi-ny dzie-dzi-na rów-nych rów-ny
	pole-ga zmie-nna pa-ra-met-rów wzo-rem po-cho-dzi
	o-trzy-ma wte-dy wa-run-ko-wych lo-gicz-nie
	skreś-la-na skreś-la-ną cał-ko-wi-tych wzo-rów po-rzą-dek po-rząd-kiem
	przy-kład pod-zbio-rów po-mię-dzy re-pre-zen-to-wa-ne
	rów-no-waż-ne bi-blio-te-kach wy-pro-wa-dza ma-te-ria-łów
	prze-ka-za-nym skoń-czo-nym moż-esz na-tu-ral-na cią-gu tab-li-cy
	prze-ka-za-nej od-po-wied-nio}

% ***************************************************************************************************
% Dokument
% ***************************************************************************************************

\frontmatter

\begin{document}

    \pagestyle{empty}
	\begin{titlingpage}
		\vspace*{\fill}
		\begin{center}
			\begin{picture}(430,500)
				\put(60,590){\makebox(0,0)[l]{\huge \textbf{Politechnika Wrocławska}}}
				\put(40,565){\makebox(0,0)[l]{\Large \textbf{Wydział Informatyki i Telekomunikacji}}}
                \put(0,550){\line(1,0){430}}
                \put(0,510){\makebox(0,0)[l]{\large Kierunek: \textbf{INF-PPT}}}
                %\textbf{3 literowy kod kierunku}}}
                \put(0,490){\makebox(0,0)[l]{\large Specjalność: \textbf{-}}}                
                %\textbf{3 literowy kod specjalności}}}                
				\put(0,370){\begin{minipage}{0.9\textwidth}
				\centering
				\Huge \textsc{Praca Dyplomowa\\ Inżynierska}
                \end{minipage}
				}
% Tytuł pracy
				\put(0,230){\begin{minipage}{0.9\textwidth}
				\centering
				\LARGE \textbf{Aplikacja rozwiązująca nonogramy}
                \end{minipage}
				}
% Autor pracy
				\put(0,170){\begin{minipage}{0.9\textwidth}
				\centering
				\Large {
				Mateusz Wałejko
				}
				\end{minipage}
				}
% dane promotora
				\put(0,90){\begin{minipage}{0.9\textwidth}
				\centering
				\large{
				Opiekun pracy\\
				\textbf{dr Marcin Michalski}
				}
				\end{minipage}
				}
				\put(0,-30){
				\begin{minipage}{0.9\textwidth}
				\normalsize{
				Słowa kluczowe: nonogram, solver, aplikacja
				}
				\end{minipage}
				}
                \put(0,-80){\line(1,0){430}}
				\put(155,-100){\makebox(0,0)[bl]{\large \textsc{Wrocław 2022}}}
			\end{picture}
		\end{center}	
		\vspace*{\fill}
	\end{titlingpage}
	
    \cleardoublepage
	% \input{streszczenie.tex}
	% \vspace*{1cm}	
    % \input{abstract.tex}
    % \cleardoublepage
	\pagenumbering{Roman}
	\pagestyle{tableOfContentStyle}
	\tableofcontents

	% spis rysunków (opcjonalnie)    
	\clearpage
	\pagestyle{listOfPlotsStyle}
        \listoffigures
        \addcontentsline{toc}{chapter}{Spis rysunków}
        
        % spis tabel (opcjonalnie)
        \clearpage
        \renewcommand{\listtablename}{Spis tabel}    
        \pagestyle{listOfTablesStyle}
	\listoftables
	\addcontentsline{toc}{chapter}{Spis tabel}




    \cleardoublepage
    
		
	% ***************************************************************************************************
	% Wstęp
	% ***************************************************************************************************
	
	\pagestyle{custom}
	\mainmatter
	
	% ***************************************************************************************************
	% Rodziały
	% ***************************************************************************************************

	%Korekta ALD - nienumerowany wstęp
%\chapter{Wstęp}
\addcontentsline{toc}{chapter}{Wstęp}
\chapter*{Wstęp}

\thispagestyle{chapterBeginStyle}



	Praca poświęcona jest aplikacji rozwiązującej nonogramy. W pracy opisane są zagadnienia
teoretyczne potrzebne do zrozumienia problemu, układ i sposób użycia samej aplikacji, a także
porównanie podejść do rozwiązywania nonogramów.

	Praca podzielona jest na 7 części.

	W pierwszej części opisany jest problem rozwiązywania nonogramów. Wytłumaczone jest
czym są nonogramy. Podane są definicje potrzebne do zrozumienia problemu. Na końcu, za pomocą
opisanych zagadnień wyznaczona jest trudność problemu.

	Druga część zawiera schematyczny opis aplikacji - listę wymagań spełnianych przez
aplikację oraz zarys nawigacji po niej.

	W trzecim rozdziale zawarte są informacje dotyczące implementacji: użyte technologie wraz
z ich opisem oraz schemat wykorzystanej bazy danych.

	Czwarta część skupia się na opisie solvera wykorzystanego w aplikacji. Nakreślona jest ewolucja
solvera w trakcie rozwoju aplikacji, zaproponowane są także heurystyki mogące usprawnić działanie
solvera. Różne implementacje wraz z usprawnieniami są poddane testom, by ocenić ich wydajność,
oraz podane są wnioski wyciągnięte z badań.

	Piąty rozdział to krótka instrukcja dla użytkownika, opisująca wymagania do uruchomienia aplikacji.

	Na końcu, w szóstej części, zawarte jest podsumowanie pracy oraz kierunki w których
można dokonać dalszego rozwoju aplikacji.

	\cleardoublepage

	\chapter{Analiza problemu}
\thispagestyle{chapterBeginStyle}
\label{rozdzial1}

    W tym rozdziale przedstawiona jest definicja nonogramów oraz sposób ich rozwiązywania.
W kolejnej części rozdziału wypisane są także definicje potrzebne do zrozumienia złożoności problemu,
jakim jest rozwiązywanie nonogramów.


\section{Przedstawienie nonogramów}

\subsection{Opis}
    Nonogramy (znane także jako \textit{Paint by Number} oraz \textit{Picross}) to łamigłówki, w których
celem jest odpowiednie wypełnienie komórek na siatce tak, by uzyskać określony wzór (np. obrazek).
W tym celu należy kolorować pola zgodnie ze wskazówkami umieszczonymi obok każdego wiersza oraz kolumny
na siatce. Wskazówki mają postać ciągu liczb - każda z liczb oznacza ilość wypełnionych komórek z rzędu,
a pomiędzy grupami wypełnionych komórek znajduje się przynajmniej jedna pusta komórka.\\
    Uściślając powyższą, nieformalną definicję, nonogram to łamigłówka na siatce wielkości $w \times h$,
gdzie $w$ oznacza szerokość planszy wyrażoną w ilości komórek, a $h$ wysokość planszy wyrażoną w
ilości komórek. Dla każdego wiersza i kolumny mamy przedstawiony ciąg liczb $H_n$ będący wskazówką
dla danej linii. Dany element $h_i$ opisuje blok stworzony z $h_i$ wypełnionych komórek z rzędu, i
jeśli $h_i$ nie jest pierwszym elementem ciągu, to blok opisany przez $h_i$ jest oddzielony 
przynajmniej jedną pustą komórką od bloku opisanego przez $h_{i-1}$, 
oraz jeśli $h_i$ nie jest ostatnim elementem ciągu, to blok opisany przez $h_i$ jest oddzielony
przynajmniej jedną pustą komórką od bloku opisanego przez $h_{i+1}$. Linię spełniającą zadaną wskazówkę
opisuje wyrażenie regularne: $l=\text{\^{}}0^*1^{h_1}0^+1^{h_2}0^+\ldots0^+1^{h_n}0^*\text{\$}$, gdzie 
$0$, $1$ oznaczają odpowiednio pustą i wypełnioną komórkę, $h_i$ jest $i$-tym elementem ciągu $H_n$,
będącego wskazówką dla wiersza/kolumny, a $|l| = n$, $n = w$ dla wiersza i $n = h$ dla kolumny. 
Rozwiązaniem nonogramu jest wypełnienie zadanej planszy w taki sposób, 
by dla każdego wiersza oraz każdej kolumny wskazówki dla nich były spełnione.

\begin{figure}[!htb]
    \centering
    \begin{subfigure}[b]{0.3\textwidth}
        \centering
        \includegraphics[width=\textwidth]{images/nonogram_example_empty.png}
        \caption{przed rozwiązaniem}
    \end{subfigure}
    \begin{subfigure}[b]{0.3\textwidth}
        \centering
        \includegraphics[width=\textwidth]{images/nonogram_example_filled.png}
        \caption{po rozwiązaniu}
    \end{subfigure}
    \caption{Przykładowa plansza}
\end{figure}

\subsection{Definicje}
\begin{definition}
    Wskazówką nazywamy ciąg $H_i$ liczb, opisujący ułożenie wypełnionych komórek w danej linii.
\end{definition}
\begin{remark}
    Dla uproszczenia opisu algorytmów następuje założenie, że pusta linia jest opisywana przez
wskazówkę będącą ciągiem pustym.
\end{remark}
\begin{definition}
    Instancją problemu nonogramu jest czwórka $N = (w, h, R_n, C_n)$, gdzie $w$, $h$ to odpowiednio
szerokość i wysokość planszy, a $R_n$ i $C_n$ są ciągami wskazówek dla wierszy oraz kolumn.
\end{definition}


\section{Wymagane zagadnienia matematyczne}

\subsection{Problem decyzyjny}
\begin{definition}
    Problem decyzyjny to problem, na który odpowiedź stanowi 'tak' lub 'nie'.
\end{definition}
    Problem decyzyjny to pojęcie kluczowe dla klasyfikacji problemu do wybranej klasy złożoności.
By móc zaklasyfikować wybrany problem (np. rozwiązanie nonogramu) do jakiejś klasy złożoności,
należy przedstawić go w postaci problemu decyzyjnego.
\begin{example}
    Czy zadany nonogram $N = (w, h, R_n, C_n)$ ma rozwiązanie?
\end{example}

\subsection{Klasa złożoności P}
\begin{definition}
    Klasa złożoności P zawiera wszystkie problemy decyzyjne, których rozwiązanie można znaleźć w czasie wielomianowym.
\end{definition}
\begin{example}
    Znalezienie najkrótszej ścieżki między dwoma punktami w grafie należy do klasy P, ponieważ
algorytm Dijkstry znajduje najkrótszą ścieżkę w czasie wielomianowym. Sformułowanie tego problemu
w postaci problemu decyzyjnego mogłoby brzmieć następująco: Czy dla danego wejściowego grafu $G = (V, E)$
istnieje ścieżka z punktu $v_1 \in V$ do punktu $v_2 \in V$ o długości niewiększej niż $x$?
\end{example}

\subsection{Klasa złożoności NP}
\begin{definition}
    Klasa złożoności NP zawiera wszystkie problemy decyzyjne, których rozwiązanie dla odpowiedzi pozytywnej
można zweryfikować w czasie wielomianowym.
\end{definition}
\begin{example}
    Mając zbiór $I$ przedmiotów, gdzie przedmiot $i_n$ to dwójka $(v_n, w_n)$, gdzie $v_n$ to wartość,
a $w_n$ to waga, oraz ograniczenie górne na sumę wag wybranych przedmiotów $w_{max}$, czy można
wybrać przedmioty w taki sposób by nie przekroczyć limitu wagi $w_{max}$, a by suma wartości
wybranych przedmiotów była większa lub równa $c$?\\
    Tak zadany problem to wersja decyzyjna problemu plecakowego. Być może nie istnieje algorytm znajdujący
przydział przedmiotów w czasie wielomianowym, ale mając przedstawione rozwiązanie $S \subseteq I$
można zsumować wartości przedmiotów z $S$ i sprawdzić czy jest to poprawne rozwiązanie.
\end{example}
    Należy zauważyć, że każdy problem z klasy \textit{P} należy także do klasy \textit{NP}, ponieważ
rozwiązanie problemu decyzyjnego jest jednym ze sposobów weryfikacji poprawności jego rozwiązania.
To czy $P = NP$ jest jak do tej pory nierozwiązanym problemem.

\subsection{Redukcja wielomianowa}
\begin{definition}
    Problem A jest redukowalny do problemu B w czasie wielomianowym, jeśli wejścia dla problemu A
można przekształcić na wejścia dla problemu B w czasie wielomianowym, a następnie rozwiązać problem A
wywołując procedurę rozwiązującą problem B wielomianową ilość razy.
\end{definition}
\begin{example}
    Można zdefiniować mnożenie liczb $a \cdot b$ za pomocą operacji dodawania w następujący sposób:
$$a \cdot b = \underbrace{a + a + \ldots + a}_{b}$$
\end{example}
    Należy zauważyć, że jeśli problem \textit{A} jest redukowalny do problemu \textit{B} w czasie wielomianowym,
a problem \textit{B} należy do klasy P, to także problem \textit{A} należy do klasy P, jako że
sposobem na jego rozwiązanie jest użycie redukcji wielomianowej by traktować go jako instancję problemu \textit{B},
a następnie rozwiązanie go za pomocą algorytmu działającego w czasie wielomianowym.
\begin{corollary}
    Jeśli istnieje redukcja z \textit{A} w \textit{B} w czasie wielomianowym, to \textit{B} jest
co najmniej tak złożony jak \textit{A}.
\end{corollary}

\subsection{Klasa problemów NP-trudnych}
\begin{definition}
    Problem \textit{H} należy do klasy problemów NP-trudnych, jeśli każdy problem w klasie NP 
jest redukowalny do \textit{H} w czasie wielomianowym.
\end{definition}
    W przypadku klasy problemów NP-trudnych nie ma wymogu, by należące do niej problemy były
problemami decyzyjnymi.
\begin{example}
    Przykładem problemu NP-trudnego jest problem spełnialności (SAT): 'Czy dla danej formuły logicznej
istnieje wartościowanie, dla którego zadana formuła jest spełniona?'. Przynależność tego problemu
do tej klasy została udowodniona w 1971 roku przez Stephena Cooka i Leonida Levina w dowodzie
twierdzenia Cooka-Levina \cite{Cook-Levin}.
\end{example}
    Dla udowadniania przynależności problemu do tej klasy kluczowa jest obserwacja, że istnienie
redukcji wielomianowej z \textit{A} w \textit{B} implikuje przynależność \textit{B} do tej klasy,
o ile \textit{A} także do niej należy.

\subsection{Klasa problemów NP-zupełnych}
\begin{definition}
    Problem decyzyjny \textit{C} należy do klasy problemów NP-zupełnych, jeśli należy do klas problemów
NP-trudnych oraz NP.
\end{definition}
\begin{corollary}
    Pokazanie, że problem decyzyjny \textit{A} jest NP-zupełny sprowadza się do pokazania, że istnieje redukcja
wielomianowa z problemu \textit{H} z klasy problemów NP-trudnych, oraz że rozwiązanie problemu \textit{A}
można zweryfikować w czasie wielomianowym.
\end{corollary}


\section{Przypisanie problemu rozwiązania nonogramów do odpowiedniej klasy złożoności}

    Mając zdefiniowane pojęcia potrzebne do klasyfikacji problemu do odpowiedniej klasy złożoności,
należy znaleźć klasę do jakiej należy rozwiązywanie nonogramów. Z uwagi na specyfikę klas, klasyfikacji
poddana zostanie decyzyjna wersja problemu, tj. 
'\textit{Czy zadany nonogram $N = (w, h, R_n, C_n)$ ma rozwiązanie?}'.

\subsection{Problem rozwiązania nonogramu jest w NP}
    Niech $M_{h, w}$ będzie macierzą oznaczająca rozwiązanie zadanego nonogramu $N = (w, h, R_n, C_n)$,
gdzie $m_{i, j}$ oznacza stan komórki w wierszu $i$ i kolumnie $j$, oraz $m_{i, j} = 1$, jeśli komórka
jest wypełniona, a $m_{i, j} = 0$ jeśli pusta. Macierz $M_{h, w}$ jest mapowana na $h + w$ list,
będących ciągami stanów komórek w kolejnych wierszach, i kolumnach.\\
    Do weryfikacji rozwiązania użyjemy następującej procedury:

\begin{pseudokod}[H]
    %\SetAlTitleFnt{small}
    \SetArgSty{normalfont}
    \KwIn{Lista linii $L$, Lista wskazówek $LH$, długość linii $n$}
    \KwOut{Poprawność rozwiązania w osi (\texttt{true/false})}
    \For{$i \leftarrow 1$ \KwTo $|L|$}{
        $Li \leftarrow L[i]$\;
        $Hi \leftarrow LH[i]$\;
        $a \leftarrow 0$\;
        $A \leftarrow []$\;
        \For{$c \in Li$} {
            \If{$c = 1$} {
                $a \leftarrow a + 1$\
            }
            \Else{
                \If{$a > 0$} {
                    $A.push(a)$\;
                    $a \leftarrow 0$\;
                }
            }
        }
        \If{$a > 0$} {
            $A.push(a)$\;
        }
        \For{$j \leftarrow 1 \KwTo |Hi|$} {
            \If{$A[j] \neq Hi[j]$} {
                \texttt{return false}\;
            }
        }
    }
    \texttt{return true}\;
    \caption{Poprawność rozwiązania w osi}\label{alg:axisValidation}
\end{pseudokod}

    W procedurze \ref{alg:axisValidation} następuje weryfikacja rozwiązania w danej osi. Przykładowo,
wywołując procedurę \ref{alg:axisValidation} dla listy wierszy i ich wskazówek, weryfikujemy Poprawność
rozwiązania w poziomie. Weryfikacja rozwiązania następuje przez wywołanie procedury dwukrotnie,
dla wierszy oraz kolumn. Jeśli w obu przypadkach procedura zwróci \texttt{true}, to rozwiązanie jest poprawne.\\
    Czas wykonania procedury jest zależny od wielkości planszy. Zewnętrzna pętla wykonuje się
tyle razy, ile jest linii w osi ($h$ w przypadku wierszy, $w$ w przypadku kolumn). Na początku pętli
dochodzi do ekstrakcji pewnych danych do lokalnych zmiennych oraz inicjalizacji tablicy - w zależności
od jezyka użytego do implementacji, ta grupa operacji zajmuje czas stały bądź liniowy. Następnie
uruchamiana jest pierwsza wewnętrzna pętla. W tej pętli analizowane są dane w danej linii, by zmapować
układ jej komórek do wskazówki jaką reprezentuje. Złożoność operacji w każdej iteracji jest stała,
jeśli założymy że powiększenie tablicy o dodatkowy element wymaga stałego czasu - w p.p. czas wykonania
iteracji może być liniowy. Ilość wykonań tej pętli zależy od długości linii.
Po wykonaniu pierwszej pętli, w zależności od układu stanu komórek w linii,
może dojść do kolejnego powiększenia tablicy o dodatkowy element - złożoność nie przekracza liniowej.
Na końcu zewnętrznej pętli wykonywana jest druga pętla, która iteruje po elementach wskazówki zadanej
w rozwiązaniu, i porównuje ich watość do analogicznych elementów we wskazówce odtworzonej z układu linii.
Rozbieżność oznacza, że rozwiązanie nie jest prawidłowe, i procedura przedwcześnie zakańcza wykonanie.
Długość wskazówki można z góry ograniczyć przez $\lceil \frac{x}{2} \rceil$, gdzie $x$ jest długością
linii.\\
    Zadana procedura sprawdza poprawność rozwiązania nonogramów, a jej złożoność, w zależności od
implementacji operacji na tablicach, może wynosić $\mathcal{O}(n^2)$ bądź $\mathcal{O}(n^3)$.
Zaproponowana procedura ma złożoność wielomianową, zatem problem decyzyjny rozwiązywania nonogramów
należy do klasy \textit{NP}.

\subsection{Problem rozwiązania nonogramu jest NP-trudny}
    Dowód NP-trudności rozwiązywania nonogramów jest obszerny i wykracza poza zakres tej pracy.
Przykładowy dowód jest opisany w pracy \cite{Nonograms-NP-Hard} i jego zarys jest następujący.
Autor rozpoczyna dowód od powołania się na NP-trudność gry na grafach, nazwanej jako
\textit{Bounded Nondeterministic Constraint Logic}. Następnie, poprzez redukcję, autor udowadnia 
NP-trudność zmodyfikowanej wersji gry, określonej na grafach planarnych. Po udowodnieniu tego faktu,
autor konstruuje redukcję wielomianową z planarnej \textit{Bounded Nondeterministic Constraint Logic}
w rozwiązywanie nonogramów, tym samym udowadniając ich przynależność do tej klasy problemów.

\subsection{Problem rozwiązania nonogramu jest NP-zupełny}
    Pokazawszy, że zadany problem jest w NP, oraz jest NP-trudny, pokazane zostało że problem ten
jest NP-zupełny.

	\cleardoublepage

	\chapter{Projekt systemu}
\thispagestyle{chapterBeginStyle}

    W tej części opisane zostały wymagania dla aplikacji w kontekście możliwości interakcji przez
użytkownika.



\section{Wymagania aplikacji}
    W ramach stworzonej aplikacji, użytkownik ma możliwość:
\begin{enumerate}
    \item wybrać jedną z predefiniowanych paczek łamigłówek, by przejść do wyboru łamigłówki;
    \item wybrać jedną z predefiniowanych łamigłówek w paczce, by przejść do jej rozwiązywania;
    \item rozwiązać łamigłówkę przy użyciu wyświetlanej planszy;
    \item oznaczać pola, co do których ma pewność, że są puste;
    \item zobaczyć stan łamigłówki w menu wyboru, by mógł ocenić:
    \begin{itemize}
        \item czy łamigłówka została rozpoczęta,
        \item czy łamigłówka została ukończona bez przegranej,
        \item czy łamigłówka została ukończona z przegraną;
    \end{itemize}
    \item wprowadzić łamigłówkę dla solvera za pomocą ekranu wprowadzania łamigłówki;
    \item wybrać łamigłówkę do rozwiązania dla solvera;
    \item nakazać rozwiązanie wprowadzonej łamigłówki.
\end{enumerate}

    Oprócz tego, aplikacja:
\begin{enumerate}
    \item śledzi błędy użytkownika w trakcie rozwiązywania łamigłówki i przerywa jej rozwiązywanie
w przypadku popełnienia zbyt wielu błędów;
    \item zapisuje postęp rozwiązywania łamigłówki przy wyjściu do poprzedniego ekranu.
\end{enumerate}



\section{Nawigacja pomiędzy aktywnościami}
    Aplikacja otwiera się w menu głównym. W menu głównym dostępne są dwie zakładki: zakłada użytkownika
i zakładka w solvera. W zakładce użytkownika użytkownik najpierw wybiera paczkę łamigłówek, a
następnie łamigłówkę do rozwiązania, po czym przechodzi do ekranu rozwiązywania. W zakładce solvera
dostępna jest lista wprowadzonych i predefiniowanych łamigłówek dla solvera. Użytkownik może przejść
do ekranu wprowadzania łamigłówki - by dodać kolejną łamigłówkę - bądź do ekranu automatycznego rozwiązywania,
gdzie nakazuje solverowi rozwiązanie wprowadzonej łamigłówki.

    Zależności między opisanymi aktywnościami są ukazane na diagramie.

\begin{figure}[!htb]
    \centering
    \includegraphics[width=\textwidth]{images/screens_diagram.png}
    \caption{Diagram zależności między aktywnościami}
    \label{diagAktywnosci}
\end{figure}

	\cleardoublepage
	
	\chapter{Implementacja systemu}
\thispagestyle{chapterBeginStyle}

	W tym rozdziale opisane są technologie i biblioteki użyte do stworzenia aplikacji, jak i schemat 
użytej bazy danych.



\section{Opis technologii}


\subsection{React Native}
	Aplikacja została napisana w bibliotece React Native \cite{React-Native}. Jest to framework
umożliwiający tworzenie aplikacji mobilnych, komputerowych oraz internetowych. React Native jest oparty
na bibliotece React.js \cite{React-Js}, stworzonej przez Jordana Walke, a rozwijanej przez
Meta Platforms Inc. oraz społeczność, i jej rozwój także jest nadzorowany przez tę firmę. Framework
jest dostępny na licencji MIT.

	Głównymi pojęciami potrzebnymi do tworzenia aplikacji w React Native są komponenty oraz stany.
Komponenty reprezentują nie tylko podstawowe składniki interfejsu graficznego, jak pola tekstowe czy 
przyciski, ale także zestawy składników realizujące określoną funkcję, np. plansza do gry lub
lista gier. Komponenty opierają się o stan, czyli zestaw informacji przechowywany w komponencie i
komunikowany do użytkownika za pomocą interfejsu. Framework odświeża wygląd interfejsu przy zmianie
stanu, która najczęściej następuje w wyniku interakcji użytkownika z interfejsem. Wtedy to dochodzi
do aktualizacji komponentów korzystających z danego stanu.

	Programowanie w React Native odbywa się za pomocą języka skryptowego JavaScript. Znak towarowy
należy do Oracle \cite{JS-Oracle}, standard jest utrzymywany przez ECMA \cite{JS-ECMA}, 
a uruchamiany jest na różnych silnikach rozwijanych zgodnie ze standardem 
(np. V8 \cite{JS-V8} od Google, czy SpiderMonkey \cite{JS-SpiderMonkey} od Mozilli).


\subsection{Dodatkowe biblioteki}
	Do stworzenia aplikacji zostały wykorzystane biblioteki tworzone przez społeczność. Są to
między innymi:
\begin{itemize}
	\item React Native Elements \cite{RN-Elements} - biblioteka zawierająca podstawowe komponenty 
zgodne z Material Design
	\item React Native Navigation \cite{RN-Navigation} - biblioteka obsługująca nawigację między 
aktywnościami
	\item React Native SQLite Storage \cite{RN-SQLite} - biblioteka pozwalająca korzystać z lokalnej
bazy SQLite
	\item React Native Zoomable View \cite{RN-Zooming} - biblioteka dodająca komponent obsługujący 
przybliżanie ekranu
\end{itemize}


\section{Baza danych}
	Dane aplikacji są przechowywane w bazie danych opartej na systemie SQLite. Schemat bazy jest
przedstawiony na diagramie poniżej.

\begin{figure}[!htb]
    \centering
    \includegraphics[width=\textwidth]{images/db_diagram.png}
    \caption{Diagram bazy danych}
\end{figure}

	Baza składa się z 3 tabel: paczek, łamigłówek i łamigłówek solvera. Paczki zawierają jedynie
niezbędne dane do reprezentacji grupy łamigłówek, czyli nazwa i ikona. W tabeli łamigłówek zawarte
są proste dane takie jak nazwa, maksymalna ilość żyć (ilość błędów, po których się przegrywa).
Przechowywane są tam również informacje o stanie gry, tj. \texttt{statusRozwiązania} - łamigłówka
nierozpoczęta i nierozwiązana, łamigłówka rozpoczęta, łamigłówka rozwiązana - oraz \texttt{typUkończenia} -
łamigłówka nieukończona bez przegranej, łamigłówka nieukończona z przegraną, łamigłówka ukończona
bez przegranej, łamigłówka ukończona z przegraną. Pod zmienną \texttt{pola} przechowywana jest lista
pól wraz z ich stanami, pod postacią stringa w formacie JSON. Tabela łamigłówek solvera także zawiera
łamigłówki, ale w innej formie - zapisane jedynie jako listy wskazówek w formacie JSON oraz dane
do identyfikacji, np. nazwa czy wielkość planszy.

	\cleardoublepage
	
	\chapter{Solver nonogramów}
\thispagestyle{chapterBeginStyle}

    W tym rozdziale opisany jest rozwój solvera. Opisane są kolejne główne wersje solverów, zbadany
jest także wpływ zastosowanych heurystyk na wydajność w rozwiązywaniu wybranych łamigłówek.



\section{Wersje solverów}


\subsection{Solver całościowy}
    Solver całościowy jest najprostszym z solverów implementowanych w toku pisania aplikacji. Jego
implementacja opiera się na założeniu, że obrazek ukryty w łamigłówce jest ciągiem pustych i pełnych
pikseli. Solver sprawdza wszystkie możliwe kombinacje pól, aż do wykrycia rozwiązania, bądź stwierdzenia
jego braku. Wskazówki umieszczone obok planszy służą jedynie do walidacji rozwiązania, i nie są
wykorzystywane w trakcie rozwiązywania nonogramu.

    Solver ten zaczyna od pustej planszy. Następnie, dla pierwszego pola wywoływana jest rekursyjna
metoda: jeśli indeks pola mieści się w zakresie planszy, to najpierw jego status ustawiany jest na
pusty, i następuje wywołanie metody dla nastepnego indeksu, a jeśli rozwiązanie nie zostanie znalezione,
to pole jest wypełniane i ponownie dochodzi do wywołania metody na następnym polu. Jeśli indeks
wykracza poza zakres planszy, to znaczy że wszystkie pola mają ustawiony status i wywoływana jest
metoda sprawdzająca poprawność rozwiązania, podobna do tej opisanej w \ref{alg:axisValidation}.
Jeśli solver zakończy działanie zwracając \texttt{true}, to w przekazanej mu macierzy pól 
(równoznaczne z listą wierszy) znajdzie się rozwiązanie łamigłówki.

\begin{pseudokod}[H]
    %\SetAlTitleFnt{small}
    \SetArgSty{normalfont}
    \SetKwFunction{Verify}{Verify}
    \KwIn{Lista wierszy $R$, indeks pola $i$, lista wskazówek wierszy $Hr$ i kolumn $Hc$, szerokość $w$ i wysokość $h$ planszy}
    \KwOut{Czy znaleziono rozwiązanie \texttt{true/false}}
    \If{$i \geq w \cdot h$}{
        \texttt{return} \Verify{$R$, $Hr$, $Hc$, $w$, $h$}\;
    }
    \Else{
        $iWiersza \leftarrow \lceil \frac{i}{w} \rceil$\;
        $iKolumny \leftarrow i\ mod\ w$\;
        $R[iWiersza][iKolumny] \leftarrow 0$\;
        \If{\texttt{SolverCałościowy}($R, i+1, Hr, Hc, w, h$)}{
            \texttt{return true}\;
        }
        $R[iWiersza][iKolumny] \leftarrow 1$\;
        \texttt{return SolverCałościowy}($R, i+1, Hr, Hc, w, h$)\;
    }
    \caption{SolverCałościowy}\label{alg:allSolver}
\end{pseudokod}

    Złożoność czasowa tego solvera jest bardzo duża. Procedura sprawdzająca poprawność rozwiązania
może zostać wywołana $2^{w*h}$ razy, czyli inaczej $2^n$, gdzie $n$ to ilość pól na planszy. 
Wynika to z faktu, że kazde kolejne pole wymaga sprawdzenia pól poprzednich, a samo może znajdować
się w dwóch stanach, więc podwaja ilość wywołań procedury sprawdzającej.


\subsection{Solver osiowy}
    W odróżnieniu od solvera całościowego, solver osiowy korzysta ze wskazówek przy szukaniu rozwiązań.
Opiera się on na fakcie, że każda linia (wiersz lub kolumna) może znajdować się w jednym z możliwych
stanów, których liczba nigdy nie dojdzie do $2^x$, gdzie $x$ jest długością linii. Sprawdzając
rozwiązanie w tym solverze, gwarantowana jest poprawność w jednej z osi, co dodatkowo znacząco skraca
czas szukania rozwiązania.

    Solver zaczyna od pustej planszy. Przed rozpoczęciem rozwiązywania sprawdzana jest ilość wszystkich
kombinacji w danej osi (iloczyn możliwości każdej z linii), i wybierana jest oś z mniejszą liczbą
możliwości. Następnie generowane są kombinacje dla każdej z linii. Solver korzysta z rekursyjnej
metody, i ustawia pierwszą kombinację dla pierwszej linii. Następnie wywołuje metodę dla kolejnej linii,
aż do ostatniej, i wtedy weryfikuje rozwiązanie. Jeśli dla danego ustawienia w linii łamigłówka
nie ma rozwiązania, to solver przechodzi do kolejnego ustawienia i wywołuje metodę w kolejnej linii.

\begin{pseudokod}[H]
    %\SetAlTitleFnt{small}
    \SetArgSty{normalfont}
    \SetKwFunction{Verify}{Verify}
    \SetKwFunction{ApplyComb}{ApplyComb}
    \KwIn{Lista linii $L$, indeks linii $i$, lista wskazówek linii prostopadłych $H$, ilość linii $n$}
    \KwOut{Czy znaleziono rozwiązanie \texttt{true/false}}
    \If{$i = n$}{
        \Verify{$L$, $H$, $n$}\;
    }
    \Else{
        $line \leftarrow L[i]$\;
        \ForEach{$comb \in line.combinations$}{
            \ApplyComb{$line, comb$}\;
            \If{\texttt{SolverOsiowy}($L, i+1, H, n$)}{
                \texttt{return true}\;
            }
        }
        \texttt{return false}\;
    }
    \caption{SolverOsiowy}\label{alg:axisSolver}
\end{pseudokod}

    Dzięki eliminacji kombinacji sprzecznych ze wskazówkami w danej osi, procedura sprawdzania
poprawności rozwiązania jest wywoływana o wiele rzadziej niż w przypadku solvera całościowego.
O ile ilość kombinacji w linii przy rozpatrywaniu każdej komórki z osobna to $2^n$, gdzie $n$ to
długość linii, tak w przypadku rozważania poprawnych kombinacji dla linii jest ona zależna od
długości i zawartości wskazówki, i można ją ograniczyć z góry przez
${n + 1 - h} \choose h$, a $h$ to ilość wskazówek w linii. To przybliżenie jest zawyżone, ponieważ
zakłada występowanie jedynie bloków długości jeden we wskazówce. W przeciętnym przypadku, bloki
wypełnionych komórek będą dłuższe, oraz będzie ich mniej. Dodatkowo, jak zostało wspomniane na początku,
weryfikacja jest wymagana jedynie w jednej z dwóch osi, jako że konstrukcja potencjalnych rozwiązań
opiera się o zestawianie poprawnych kombinacji z linii.

	\cleardoublepage
	
	\chapter{Instalacja i wdrożenie}
\thispagestyle{chapterBeginStyle}

W tym rozdziale należy omówić zawartość pakietu instalacyjnego oraz założenia co do środowiska, w którym realizowany system będzie instalowany. Należy przedstawić procedurę instalacji i wdrożenia systemu. Czynności instalacyjne powinny być szczegółowo rozpisane na kroki. Procedura wdrożenia powinna obejmować konfigurację platformy sprzętowej, OS (np. konfiguracje niezbędnych sterowników) oraz konfigurację wdrażanego systemu, m.in.\ tworzenia niezbędnych kont użytkowników. Procedura instalacji powinna prowadzić od stanu, w którym nie są zainstalowane żadne składniki systemu, do stanu w którym system jest gotowy do pracy i oczekuje na akcje typowego użytkownika.


	\cleardoublepage
	
	\chapter{Podsumowanie}
\thispagestyle{chapterBeginStyle}

W podsumowanie należy określić stan zakończonych prac projektowych i implementacyjnych. Zaznaczyć, które z zakładanych funkcjonalności systemu udało się zrealizować. Omówić aspekty pielęgnacji systemu w środowisku wdrożeniowym. Wskazać dalsze możliwe kierunki rozwoju systemu, np.\ dodawanie nowych komponentów realizujących nowe funkcje.

W podsumowaniu należy podkreślić nowatorskie rozwiązania zastosowane w projekcie i implementacji (niebanalne algorytmy, nowe technologie, itp.).




	\cleardoublepage
	
	
	%%%%%%%%%%%%%%%%%%%%%%%%%%%%%%%%%%%%%%%%%%%%%%%%%%%%%%%%%%%%%%%%%%%%%%%%%%%%%%
	%%%%%%%%%%%%%%%%%%%%%%%%%%%%%%% BIBLIOGRAFIA %%%%%%%%%%%%%%%%%%%%%%%%%%%%%%%%%
	%%%%%%%%%%%%%%%%%%%%%%%%%%%%%%%%%%%%%%%%%%%%%%%%%%%%%%%%%%%%%%%%%%%%%%%%%%%%%%

	\nocite{*}
	\pagestyle{bibliographyStyle}
	\bibliographystyle{ieeetr}
	\bibliography{literatura}
	\thispagestyle{chapterBeginStyle}
        \addcontentsline{toc}{chapter}{Bibliografia}
	\cleardoublepage
	
	%%%%%%%%%%%%%%%%%%%%%%%%%%%%%%%%%%%%%%%%%%%%%%%%%%%%%%%%%%%%%%%%%%%%%%%%%%%%%%
	%%%%%%%%%%%%%%%%%%%%%%%%%%%%%%%%% DODATKI %%%%%%%%%%%%%%%%%%%%%%%%%%%%%%%%%%%%
	%%%%%%%%%%%%%%%%%%%%%%%%%%%%%%%%%%%%%%%%%%%%%%%%%%%%%%%%%%%%%%%%%%%%%%%%%%%%%%
	
	\appendix
	\pagestyle{appendixStyle}
       \renewcommand{\appendixname}{Załącznik}
	
	\chapter{Zawartość płyty CD}
\thispagestyle{chapterBeginStyle}
\label{plytaCD}

Na dołączonej płycie znajdują się cztery katalogi. 
W katalogu \texttt{thesis} znajduje się plik z pracą dyplomową w formie elektronicznej. 
W katalogu \texttt{thesis-src} znajdują się pliki źródłowe pracy dyplomowej.
W katalogu \texttt{nogram} znajduje się plik instalacyjny aplikacji o nazwie \texttt{nogram.apk}.
W katalogu \texttt{nogram-src} znajdują się pliki źródłowe aplikacji.

	
	\cleardoublepage

\end{document}
